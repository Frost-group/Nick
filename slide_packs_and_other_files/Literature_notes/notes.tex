\documentclass{article}
\hfuzz=1.5pt
\usepackage{fancyhdr}
\pagestyle{fancy}
\fancyfoot{}
\lhead{}
\chead{}
\rhead{}
\lfoot{}
\cfoot{}
\rfoot{{\it Nicholas Siemons, University College London}}
\renewcommand{\headrulewidth}{0pt}
\renewcommand{\footrulewidth}{0.7pt}
\parindent=0pt
\parskip=1mm
\headheight=-5pt
\headsep=0pt
\oddsidemargin=-15pt
\evensidemargin=20pt
\textheight=620pt
\textwidth=500pt
\footskip=20pt
\mathsurround=2pt
\usepackage{graphicx}
\usepackage{epstopdf}
\usepackage{chronology}
\usepackage{authblk}
\usepackage{caption}
\usepackage[version=3]{mhchem}
\usepackage{color}
\usepackage[titles]{tocloft}
\DeclareGraphicsRule{.tif}{png}{.png}{`convert #1 `basename #1 .tif`.png}



\begin{document}

\section*{Literature Notes}



\vspace{1cm}
{\bf \cite{Ionov2014}}

Looking at using polymers to do work on micro scale.  Stimuli responsive hydrogel actuators.  Look at different stimuli and different deformations. Different types of movement etc. 

Hydrogels can be 99\% water by mass. Can swell and shrink by 10x volume. Stimuli can be light, pH, ionic strength and applied voltage.  Swelling controlled by the free-energy for the network expansion. This depends on cross-linking density and molar free energy of mixing. That depends on interactions between polymer and solvent as well as mixing entropy.  (Flory-Rehner Theory) - worth looking into. 

Actuation can be done using a Belousov-Zhabotinsky redox reaction - look into.  See \cite{Suzuki2012,Maeda2010}.  

Types of Deformation.  Typically uniform deformation. Bending and twisting etc due to inhomogeneous expansion.  Can apply inhomogeneous field to achieve.  Or can use inhomogeneous hydrogel, eg layered. 

Shape of hydrogel - rods or layers.  Rods - bending of rods can be achieved with bilayers. Always bends in direction of non-swollen side (makes sense).





\vspace{1cm}
{\bf \cite{Strakosas2015}}

About OECT for biological sensing.  Attractive due to biocompatibility (??) and mechanical matching between sensor and environment.  OECT has high amplification and can therefore be a high-fidelity transducer.  

Needs of the community - detecting low conc analytes, low voltage brain activity, and pathogens, and improving biocompatibility.  

See \cite{White1984} for first reported OECT. 

See \cite{blaudeck2012simplified}, \cite{basirico2011inkjet}, \cite{basirico2012electrical}, \cite{pettersson2014patterned}, \cite{nilsson2002all} and \cite{kawahara2013fast} for manufacturing (cheap).

See \cite{fraser2011electrochemical} and \cite{shim2009all} for sensing of glucose is saliva. 

See \cite{sessolo2013simple} for miniaturisation and neuron interfacing. 

See \cite{mabeck2005microfluidic} for lab-on chip technology. 

See \cite{leleux2011highly} for potential for long-tern and non-invasive recording.

See \cite{hamedi2007towards}, \cite{mattana2011organic} and \cite{muller2011woven} for integration to fabric. 

Nice introduction to CP's as a whole - worth looking at.  Briefly describes why they might be good for biocompatibility.  Goes on to talk about PEDOT:PSS.  PEDOT doped with PSS.  It is a p-type CP.  PSS has negative charge, and PEDOT backbone has positive charge.  high electronic conductivity, approx 1000 S/cm.  

See \cite{Bernards2007} for theory on OECT and effective circuit. 








 \vspace{1cm}
{\bf \cite{Giovannitti2016}}

Motivation partly to develop OECT in the accumulation mode (when voltage is zero, no charge flows). Propose that the glycol side chains help with hydration and ion penetration.  

OECTs - electrolyte can be solid, gel or liquid as long as ions can rearrange when the gate voltage is applied.   

They use p(a2T-TT) and p(g2T-TT), as they say that the optoelectronic proeprties do not change much with different sidechains.  They say the oxy group has planarising effect, and cite the paper \cite{Guo2013}.  Worth checking this out. 

p(g2T-TT) has slightly lower ionisation potential.  They have similar absorption, also shown in DFT. Roughness and Surface morphology shown to be similar (AFM).  

Material swelling correlated to device performance, see \cite{Inal2016} and \cite{Stavrinidou2013}.  p(g2T-TT) shows 10\% swelling when exposed to water, and further 10\% with Cl- ions under voltage. 




 \vspace{1cm}
{\bf \cite{Giovannitti2018}}

Looking at n-type polymers.  Look at polymers with different percentage of alkyl or glycol side chains.   They discuss a little bit the effect of changing the alkyl for glycol on the dominant forces acting inter and intra monomer. alkyl is predominantly dispersion,  glycol is electrostatic.

Glycol can decrease pi-pi stacking distance and increase dielectric constant. 

See \cite{Chen2016} and \cite{Meng2015} which suggests glycol reduces pi pi stacking (different polymer series).  

See \cite{Strakosas2015} and \cite{Rivnay2018} to see about OECTs in biological events.  

Read about PEDOT:PSS depletion mode OECT.  \cite{Khodagholy2013}, \cite{Khodagholy2013a} and \\\cite{Campana2014}.  

Glycol side chains decreases ionisation potential slightly from 5.7eV at 0\% glycol to 5.5eV at 100\% glycol. Suggests glycol side chain might interact with the polymer backbone and increase electron density of the copolymer - interesting!!

Again use transconductance as figure of merit. P90 and P100 show best performance.  

Polarity of polymers increases significantly as you have more glycol chains, as expected. 

P0 to P50 show about 10\% swelling. P75 is 12\%, P90 is 42\% and P100 is 102\%.  

They also compare this with other studies and suggest the length of the glycol side chain has a significant effect on the swelling behaviour.  

Reduction potentials shift from -1.12V for P0 to 0.24V for P100, and as they have the same backbone it is likely due to ion penetration into the bulk.  So if you have better ion penetration, you have a lower reduction potential - makes sense as the electrons can more easily go from the polymer to the ion.





 \vspace{1cm}
{\bf \cite{Gladisch2019}}  

 Looking predominantly at volume changes of polymers.  For reversible systems they say its about 40\%, and for irreversible its 100\%.  Needs to be higher.  High amounts of swelling can occur when you switch from a solid state to a gel reversibly.  Need to maintain percolation.  Causes of expansion include inclusion of water, ions and polymer rearrangement. 

Briefly discuss polypyrrole which is so far the best swelling conjugated polymer \cite{carpi2009biomedical} (40\% swelling). Good due to porous structure. However backbone is hydrophobic, which isn't ideal.   

Hydrogels - see \cite[Ionov, 2014]{Ionov2014} about hydrogels (typically non-conjugated). 

They look at volumetric switching of p(gT2).  Carbon electrode coated in p(gT2).  

In voltage range of +/- 0.8V - First cycle gets 12,000\% expansion at +0.8V.   Seems to go from solid state to gel.   When -0.8V applied, contracts inhomogeneously.  They suggest application in single-switch electro-valve applications. 

In voltage range of +/- 0.5V - perhaps reversible ('safe electrochemical window').  Initial expansion of 1400\%.  In subsequent cycles typically it has 300\% expansion. Up to 300 cycles in range +0.5 to -0.2V. 

MD - Little/no explanation of where their forcefield comes from.  They show swelling but admit that the limitations of the MD mean it does not give agreement with experiments.  They do however show that the pi-stack is maintained which is quite nice.  

 


 \vspace{1cm}
{\bf \cite{Flagg2019}}

Look at P3MEEMT, which is similar to our polymer but without the oxygen on the backbone.  Characterise OECT as a function of the anion. Compared to P3HT it has faster ion injection rates, which are attributed to the swelling of the lattice. Also look as a function of film crystallinity. Increased crystallinity increases OFET mobility, but decreases OECT mobility.  They say this is due to the aqueous environment.  They also use EQCM.   Inclusion of water can reduce the electronic connectivity between crystalline regions, lowering mobility in solution.  

Good citations for applications.  

See \cite{Inal2017} for benchmarking OECTs.  

Suggests high hole mobility is due to high crystallinity, but this tends to inhibit bulk ionic transport. \cite{Tseng2014}.  On the other hand, high capacitance is achieved in disordered media, which have poor electronic transport \cite{Rivnay2016} (also about PEDOT:PSS).  Tradeoff seen in \cite{Giridharagopal2017} - worth reading. 

See \cite{Laiho2011} and \cite{Toss2014} for carboxylic acid capped side chains.  See \cite{Pacheco-Moreno2017} for alcohol capped side chains increasing magnitude and speed of ion uptake. Largely attributed to increased hydrophilicity. 

See \cite{Paterson2018} to see about design rules for OECTs.

Looks to address that design rules need to be updated in order to account for polymer-solvent and ion interactions. 

Find that highly crystalline P3MEEMT has lower OECT mobility compared to amorphous P3MEEMT, but OFETS have the reverse trend.  





 \vspace{1cm}
{\bf \cite{Savva2019}}

When less water is injected, you get faster switching and higher transconductance. Swelling can be a bad thing in this case. 

See \cite{Inal2014} to see about accumulation mode. 

For PEDOT:PSS see \cite{inal2016optical}, \cite{volkov2017understanding} and \cite{rivnay2016structural} for PEDOT:PSS.

Changes in P3HT lattice spacings and crystal orientation on electrochemical doping \cite{guardado2017structural}.  Changes occur in the amorphous phase too \cite{giridharagopal2017electrochemical}.

Look at influence of water on doping, structure and transport properties on p(g2T-TT).  They modulate the electrolyte concentration to modulate the hydration properties of the dopant.  Can look at just the water drift in and out.  There is a sweet spot for hydration.  Too much water will impare hole mobility and therefore reduce transconductance.  In contrast to PEDOT systems, where hydration is always beneficial.  Suggest minimising the volumetric expansion is important. 





 \vspace{1cm}
{\bf \cite{Savva2020}}

Use p-type backbone.  Hydrophilic ions needed to facilitate transport.  Swelling of polymer is not de-facto beneficial for balancing ionic and electronic conduction.  Heterogeneous water uptake disrupts electronic conduction. Leads to OECT with slower response and worse transconductance. 

Suggest from previous PEDOT:PSS studies that the ion-electron coupling depends significantly on microstructure, morphology and arrangement in the solid state. 

Recent studies have shown that performance affected by microstructure and nanostructure \cite{Savva2020}, \cite{Flagg2019}.  Especially second, as it is about polymer crystallinity.  

They look at p(aT-TT) and p(gT-TT), but they also look at another polymer with 6 glycol units on the side chain.  They say it is interesting because it is more hydrophilic and therefore may yield even higher transconductance.   It however has lower transconductance. 

g-100\% switch order of magnitude faster than g-75\% and lower, $\tau = 116 \mu s$.  

Redox behaviour essential.  Can be explored via cyclic voltammettry. more glycol causes oxidation potential to go closer to zero (polymer losing electrons).  

Also show that swelling is correlated with glycol units.  Use EQCM-D.   

Excessive hydration can inhibit hole transport.  

They say the hexagisglycol polymer had worse performance due to the uneven volumetric expansion of the polymers microstructure due to excessive water uptake. Could be that percolation networks are disrupted due to excessive water uptake.   


 \vspace{1cm}
{\bf \cite{Schmode2020}}

polythiophene homopolymers. No spacer, methyl and ethyl spacer and glycol side chains. Test in OECTs. Very different performances.  Highest $\mu C*$ shown for 11.5 F/cmVs for ethyl spacer. Investigate crystal structure in the dry state using WAXS and GIWAXS.  Water uptake also studied. When attached with no ethyl linker it is amorphous.  with longer ethyl spacers you have more crystallinity. Also the ethyl linker causes increased swelling.  

The bigger the spacer, the greater the crystallinity. Also it shows the best OECT performance. 











\newpage

\footnotesize
% \nocite{*}
\bibliographystyle{apalike}
  \bibliography{notes}
\end{document}




